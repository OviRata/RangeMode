% This is samplepaper.tex, a sample chapter demonstrating the
% LLNCS macro package for Springer Computer Science proceedings;
% Version 2.21 of 2022/01/12
%
\documentclass[runningheads]{llncs}
%
\usepackage[T1]{fontenc}
% T1 fonts will be used to generate the final print and online PDFs,
% so please use T1 fonts in your manuscript whenever possible.
% Other font encondings may result in incorrect characters.
%
\usepackage{graphicx}

\usepackage{amsmath}
\usepackage{multirow}
% Used for displaying a sample figure. If possible, figure files should
% be included in EPS format.
%
% If you use the hyperref package, please uncomment the following two lines
% to display URLs in blue roman font according to Springer's eBook style:
%\usepackage{color}
%\renewcommand\UrlFont{\color{blue}\rmfamily}
%
\begin{document}
%
\title{Linear-Space Data Structure for Range Mode with Logarithmic Query Time}
%
%\titlerunning{Abbreviated paper title}
% If the paper title is too long for the running head, you can set
% an abbreviated paper title here
%
\author{Ovidiu Rața\inst{1}, Paul Flavian Diac\inst{1}}
%
% First names are abbreviated in the running head.
% If there are more than two authors, 'et al.' is used.
%
\institute{Alexandru Ioan Cuza Universy of Iași,\\ Faculty of Computer Science, Iași, Romania}
%
\maketitle              % typeset the header of the contribution
%
\begin{abstract}
The abstract should briefly summarize the contents of the paper in
150--250 words.

\keywords{Data structure  \and Logarithmic Time \and Range queries \and Mode \and Linear-Space.}
\end{abstract}
%
%
%
\section{Introduction}


\section{Data Structure Construction}

\subsection*{Data Structure Precomputation.}

\subsection*{Additional Definitions.}
For an array $A$, an element $x$ and an interval $[i,j]$, we will define $A.cnt_x(i,j)$, 
as the number of occurrences of element $x$ in the range $A[i:j]$. 
In terms of the fundamental \textbf{rank}$\backslash$ \textbf{select} operations: $A.cnt_x(i,j)=\textbf{rank}_x(A,j)-\textbf{rank}_x(A,i-1)$

For any block $i$ , let $L(i)$ be its left endpoint, and $R(i)$ be its right endpoint.
For any pair of blocks $i$, $j$ s.t. $i\leq j$, let $\phi(i,j)$ be the frequency of the range mode in the range $A[L(i):R(j)]$, 
and let $\mu(i,j)$ be the range mode value, calculated by the precomputation algorithm, for the range $A[L(i):R(j)]$.

We define the function $\phi'$ the following way: 
\[
     \phi'(i,j)=
     \begin{cases} 
        \phi(i,j)-\phi(i,j-1) , & j > i \\
        \phi(i,i) , &
    \end{cases}
\]
We define the function $\phi''$ the following way:
\[
    \phi''(i,j)=
    \begin{cases}
        \phi'(i,j)-\phi'(i-1,j) , & i>1 \\
        \phi'(i,j), & i=1
    \end{cases}
\]



\begin{lemma}
    For any pair of blocks $i$, $j$ s.t. $j\geq i$, the following holds: 
    \[
        \phi'(i,j)\geq 0
    \]
\end{lemma}
\begin{proof}
    
\end{proof}

\begin{lemma}
    For any pair of blocks $i$, $j$ s.t. $j\geq i$, the following holds:
    \[
        \phi''(i,j)\geq 0
    \]
\end{lemma}
\begin{proof}
    
\end{proof}

\begin{lemma}
    For any pair of blocks $i$, $j$ s.t. $j\geq i$, the following holds:
    \[
        \phi'(i,j) = cnt_{\mu(i,j)}(L(j), R(j))
    \]
\end{lemma}
\begin{proof}
    
\end{proof}


\begin{lemma}
    For any pair of blocks $i$, $j$ s.t. $j\geq i$, the following holds:
    \[
        \phi''(i,j) \leq cnt_{\mu(i,j)}(L(j), R(j))
    \]
\end{lemma}
\begin{proof}
    
\end{proof}

\begin{lemma}
    For any pair of blocks $i$, $j$ s.t. $j\geq i$, 
    if $\phi''(i,j)>0$, then for any $1\leq k<i$, the following holds:

    \[
        \text{If $\phi''(k,j)>0$, then $\mu(k,j)\neq \mu(i,j)$}
    \]
\end{lemma}
\begin{proof}
    
\end{proof}

\begin{theorem}
    \[
        \sum_{i=1}^{s}\sum_{j=i}^{s} \phi''(i,j) \leq n
    \]
\end{theorem}
\begin{proof}
    
\end{proof}


\begin{theorem}
    \[
        \phi(i,j) = \sum_{t=i}^{j} \sum_{k=1}^{i} \phi''(k,t)
    \]
\end{theorem}
\begin{proof}
    
\end{proof}

Now, we  will define the array $S$ the following way:
\begin{definition}
    The array $S$ consists of elements in the alphabet $\{1,2,\dots,s+1\}$, 
    and obeys the following properties:
    
    \begin{property}
        The array $S$ contains $s$ elements with value $s+1$. 
        The $k$-th element with value $s+1$ corresponds to the right border of the $k$-th blocks.  
    \end{property}
    
    \begin{property}
        For any $1\leq i\leq n$, for any $1 \leq k \leq i$, 
        in the interval of $S$ between the $i-1$-th element with value $s+1$ 
        and the $i$-th element with value $s+1$ there will be $\phi''(k,i)$ elements with value $k$. 
    \end{property}

\end{definition}

\begin{lemma}
    For any pair of blocks $i$, $j$, s.t. $i\leq j$, 
    let $p_1=\textbf{select}_{s+1}(S, j-1)$ and $p_2=\textbf{select}_{s+1}(S, j)$. 
    Then, the following holds:
    \[
        \phi''(i,j)=S.cnt_{i}(p_1, p_2)
    \]
\end{lemma}
\begin{proof}
    
\end{proof}


\begin{lemma}
    For any pair of blocks $i,j$, s.t. $i\leq j$,
    let $p_1=\textbf{select}_{s+1}(S, i-1)$ and  $p_2=\textbf{select}_{s+1}(S,j)$.
    Then, the following holds:
    \[
        \phi(i,j) = \sum_{k=1}^{i}S.cnt_{k}(p_1, p_2)
    \]  
\end{lemma}
\begin{proof}
    
\end{proof}






%
% ---- Bibliography ----
%
% BibTeX users should specify bibliography style 'splncs04'.
% References will then be sorted and formatted in the correct style.
%
% \bibliographystyle{splncs04}
% \bibliography{mybibliography}
%
\begin{thebibliography}{8}
\bibitem{ref_article1}
Author, F.: Article title. Journal \textbf{2}(5), 99--110 (2016)

\bibitem{ref_lncs1}
Author, F., Author, S.: Title of a proceedings paper. In: Editor,
F., Editor, S. (eds.) CONFERENCE 2016, LNCS, vol. 9999, pp. 1--13.
Springer, Heidelberg (2016). \doi{10.10007/1234567890}

\bibitem{ref_book1}
Author, F., Author, S., Author, T.: Book title. 2nd edn. Publisher,
Location (1999)

\bibitem{ref_proc1}
Author, A.-B.: Contribution title. In: 9th International Proceedings
on Proceedings, pp. 1--2. Publisher, Location (2010)

\bibitem{ref_url1}
LNCS Homepage, \url{http://www.springer.com/lncs}. Last accessed 4
Oct 2017
\end{thebibliography}
\end{document}
